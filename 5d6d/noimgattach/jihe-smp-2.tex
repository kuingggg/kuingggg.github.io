\documentclass[a4paper]{ctexart}%icle}
\usepackage[margin=2cm]{geometry}
\usepackage{amsmath,amssymb,amsthm}
\renewcommand{\baselinestretch}{1.67}

\makeatletter
\def\fnum@figure#1{\figurename\nobreakspace\thefigure}
\makeatother

\usepackage{tikz}
\usetikzlibrary{calc,intersections,through,backgrounds}

\begin{document}


如图\ref{fig1},将三角板放在正方形$ABCD$上,使三角板的直角顶点$E$与正方形$ABCD$的顶点 重合.三角板的一边交$CD$于点$F$ ,另一边$CB$ 的延长线于点$G$.


(1)求证:$EF=FG$;


(2)如图\ref{fig2},移动三角板,使顶点$E$始终在正方形 的对角线 上,其他条件不变,(1)中的结论是否仍然成立?若成立,请给予证明;若不成立,请说明理由;


(3)%如图3,
将(2)中的“正方形$ABCD$ ”改为“矩形$ABCD$ ”,且使三角板的一边经过点$B$,其他条件不变,若$AB=a,BC=b$,求$\dfrac{EF}{EG}$ 的值.


%忘记标注代码含义,以后补吧
\begin{figure}[!ht]
\begin{minipage}[b]{0.5\linewidth}
\centering
\begin{tikzpicture}[scale=2]
  \draw (-1.5,0)--(0,0)node[below] {$B$};
  \draw (0,0)--(2,0)node[below]{$C$}--(2,2)node[above]{$D$}-- (0,2)node[above]{$A(E)$}--cycle;
  \draw (0,2)--(2,0);
              \path [draw,name path=AG] (0,2)--(-1.5,-0.5);
              \path [draw,name path=BG] (0,0)--(-1.5,0);
              \path [name intersections={of=AG and BG,by=G}];
             \node[label=below:$G$] at (G){};
             \coordinate (F) at ($ (0,2)!1! 90:(G) $);
             \draw (0,2)--(F)node[right]{$F$};
\end{tikzpicture}
\caption{}
\label{fig1}
\end{minipage}%
\begin{minipage}[b]{0.5\linewidth}
\centering
\begin{tikzpicture}[scale=2]
  \draw (-1.5,0)--(0,0)node[below] {$B$};
  \draw (0,0)--(2,0)node[below]{$C$}--(2,2)node[above]{$D$}-- (0,2)node[above]{$A$}--cycle;
  \draw (0,2)--(.5,1.5)node[left]{$E$}--(2,0);
              \path [draw,name path=EG] (0.5,1.5)--(-.5,-0.5);
              \path [draw,name path=BG] (0,0)--(-1.5,0);
              \path [name intersections={of=EG and BG,by=G}];
             \node[label=190:$G$] at (G){};
             \coordinate (F) at ($ (0.5,1.5)!1! 90:(G) $);
             \draw  (0.5,1.5)--(F)node[right]{$F$};
\end{tikzpicture}
\caption{}
\label{fig2}
\end{minipage}%
\end{figure}

%\begin{tikzpicture}    %未完成 明天再说 
%  \draw (-1.5,0)--(0,0)node[below] {$B$};
%  \draw (0,0)--(2,0)node[below]{$C$}--(2,1)node[above]{$D$}-- (0,1)node[above]{$A$}--cycle;
%  \draw (0,2)--(.5,1.5)node[left]{$E$}--(2,0);
%              \path [draw,name path=EG] (0.5,1.5)--(-.5,-0.5);
%              \path [draw,name path=BG] (0,0)--(-1.5,0);
%              \path [name intersections={of=EG and BG,by=G}];
%             \node[label=190:$G$] at (G){};
%             \coordinate (F) at ($ (0.5,1.5)!1! 90:(G) $);
%             \draw  (0.5,1.5)--(F)node[right]{$F$};
%  \node at (.25,-.75) [below]{图3};
%\end{tikzpicture}



\end{document}
