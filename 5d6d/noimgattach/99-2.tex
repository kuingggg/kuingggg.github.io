\documentclass[11pt,a4paper]{ctexart}
\usepackage[margin=2cm]{geometry}
\usepackage{amsmath,amssymb,amsthm}
\renewcommand{\baselinestretch}{1.5}
\usepackage[colorlinks=true]{hyperref}
\usepackage{bookmark}
\usepackage{enumitem}
%\usepackage{esvect}%向量\vv
\pagestyle{empty}
\begin{document}
\begin{center}
\zihao{3}\heiti 几道函数经典题例析\footnote{\zihao{-5}\kaishu 适用必修一或高三第一轮复习基本初等函数I}
\end{center}
\begin{enumerate}[align=left,leftmargin=0pt,labelindent=1\parindent,listparindent=\parindent,labelwidth=0pt,itemindent=!,label= \textbf{题\arabic*:}]
\item 若集合$A=\{x\in \mathbf{R}|\dfrac{x^2-4}{x+a}=1,a\in\mathbf{R}\}$的子集有且只有两个,求实数$a$的取值集合$M$.


\vfill

\item 已知函数$f(x)=\ln (ax^2+x+1)$,根据下列条件求实数$a$的取值范围:
\begin{enumerate}[align=left,leftmargin=0pt,labelindent=0.75\parindent,listparindent=\parindent,labelwidth=0pt,itemindent=!,label= (\arabic*)]
\item 定义域为$\mathbf{R}$;
\item 值域为$\mathbf{R}$.
\end{enumerate}
\vfill

\item 已知函数$f(x)=\dfrac x{ax+b},(a,b\text{为常数,且}a\ne0)$满足$f(2)=1$,方程$f(x)=x$有唯一实数解,求函数$f(x)$的解析式,并求$f(f(-3))$的值.
\vfill
\newpage



\item 设$f(x)$是定义在$\mathbf{R}$上且周期为2的函数,在区间$[-1,1]$上,\( f(x)=\begin{cases}ax+1,&-1\le x <0,\\ \dfrac{bx+2}{x+1},&0\le x\le1, \end{cases} \)其中$a,b\in \mathbf{R}$.若$f(\dfrac12)=f(\dfrac32)$,则$a+3b$的值为\underline{\hbox to 15mm{}}.

\vfill

\item 写出一个满足满足$f(xy)=f(x)+f(y)-1$,则函数$f(x)= $\underline{\hbox to 15mm{}};若$f(x)$为单调函数,则$f(x)=$\underline{\hbox to 15mm{}}.

\bigskip
\bigskip

\item 写出一个满足$f(x+y)=2f(x)f(y)$,则函数$f(x)= $\underline{\hbox to 15mm{}}.


\bigskip
\bigskip



\item 已知三个正数$a,b,c$,满足$a^3+b^3=c^3$,那么这三个正数$a,b,c$\mbox{(\hspace{1cm})}
\begin{enumerate}[align=left,leftmargin=0pt,labelindent=\parindent,listparindent=\parindent,labelwidth=0pt,itemindent=!,label= \Alph*.]
\item 能组成一个锐角三角形的三边
\item 能组成一个直角三角形的三边
\item 能组成一个钝角三角形的三边
\item 不能组成一个三角形的三边
\end{enumerate}
\vfill

\item 若$m,n$均为实数,且$m^3-3m^2+5m=1,n^3-3n^2+5n=5$,求$m+n$的值\footnote{\zihao{-5}\kaishu 对高一,提示:利用函数奇偶性,即$m^3-3m^2+5m=(m-1)^3+2(m-1)+3,\text{\href{http://www.pep.com.cn/rjwk/gzsxsxkj/sxkj2013/sxkj11/sxkj11gk/201301/t20130130_1148979.htm} {构造奇函数}}f(x)=x^3+2x$}.
\vfill
\end{enumerate}
\end{document}